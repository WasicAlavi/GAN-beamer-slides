% ============================================================
%  Section 3: GAN Structure
% ============================================================
\section{GAN Structure}
% ── Colour guard (safe duplicate – only defines if not yet set) ──────────────
\providecolor{cream}{RGB}{250,248,240}
\providecolor{crimson}{RGB}{139,26,26}
\providecolor{darkfoot}{RGB}{60,60,60}
\providecolor{lightcrimson}{RGB}{200,60,60}
\providecolor{blockbg}{RGB}{242,236,224}


% ── Slide 1: Overview ──────────────────────────────────────────────────────
\begin{frame}{Two Players: Generator \& Discriminator}

% --- Definition box on top ---
\begin{block}{Structure}
A \textbf{Generative Adversarial Network (GAN)} consists of two neural networks trained together:
a \textbf{Generator} $G$ that creates synthetic samples, and a \textbf{Discriminator} $D$ that
distinguishes real samples from fake ones.
\end{block}

\vspace{0.2em}

% --- Side-by-side boxes: Generator and Discriminator ---
\begin{columns}[T,onlytextwidth]
  \begin{column}{0.5\textwidth}
    \begin{tcolorbox}[title=Generator $G$, equal height group=A, colback=blockbg]
      \begin{itemize}\itemsep0.45em
        \item \textbf{Input:} random noise $z$
        \item \textbf{Output:} synthetic sample $G(z)$
        \item \textbf{Goal:} fool $D$
      \end{itemize}
    \end{tcolorbox}
  \end{column}
  \begin{column}{0.5\textwidth}
    \begin{tcolorbox}[title=Discriminator $D$, equal height group=A, colback=blockbg]
      \begin{itemize}\itemsep0.45em
        \item \textbf{Input:} real $x$ or fake $G(z)$
        \item \textbf{Output:} real/fake score
        \item \textbf{Goal:} catch fakes
      \end{itemize}
    \end{tcolorbox}
  \end{column}
\end{columns}
\end{frame}

% ── Slide 1: Early Training ────────────────────────────────────────────────
\begin{frame}{Training Intuition: Early Stage}
\centering
\includegraphics[width=\linewidth,height=0.8\textheight,keepaspectratio]{bad_gan}

\vspace{1.3em}

\begin{itemize}
  \item Generator produces obvious fakes; discriminator easily detects them.
\end{itemize}

\end{frame}


% ── Slide 2: Mid Training ──────────────────────────────────────────────────
\begin{frame}{Training Intuition: Mid Stage}
\centering
\includegraphics[width=\linewidth,height=0.8\textheight,keepaspectratio]{ok_gan}

\vspace{1.3em}

\begin{itemize}
  \item Generator improves and begins to fool the discriminator.
\end{itemize}

\end{frame}


% ── Slide 3: Late Training ─────────────────────────────────────────────────
\begin{frame}{Training Intuition: Late Stage}
\centering
\includegraphics[width=\linewidth,height=0.8\textheight,keepaspectratio]{good_gan}

\vspace{1.3em}

\begin{itemize}
  \item Discriminator struggles to distinguish real from fake; accuracy drops.
\end{itemize}

\end{frame}

%------------ whole system
\begin{frame}{GAN Architecture: Full System Overview}

\centering
\includegraphics[width=\linewidth,height=0.8\textheight,keepaspectratio]{gan_diagram}
\end{frame}

% ── Slide 2: Discriminator Deep-Dive ───────────────────────────────────────
\begin{frame}{The Discriminator in a GAN}
\small

\begin{block}{What is the Discriminator?}
The discriminator $D$ is a \textbf{classifier} that learns to distinguish:
\begin{itemize}
    \item Real data from the dataset
    \item Fake data produced by $G$
    \item $D$ can use any suitable architecture (e.g., CNN for images)
\end{itemize}
\end{block}
\end{frame}


\begin{frame}{Discriminator Training}
\small

\begin{block}{Training Data \& Update Process}
\begin{columns}[T,onlytextwidth]

\begin{column}{0.48\textwidth}
\textbf{Training Data}
\begin{itemize}
    \item Real samples → positive
    \item $G(z)$ → negative
\end{itemize}
\end{column}

\begin{column}{0.48\textwidth}
\textbf{Update Step}
\begin{itemize}
    \item Classify real \& fake
    \item Compute loss
    \item Update \textbf{only $D$}
\end{itemize}
\end{column}

\end{columns}
\end{block}

\end{frame}



\begin{frame}{Discriminator Training: Backpropagation}

\centering
\includegraphics[width=0.9\linewidth,height=0.8\textheight,keepaspectratio]{gan_diagram_discriminator}

\end{frame}



% ── Slide 3: Generator Deep-Dive ───────────────────────────────────────────
\begin{frame}{The Generator in a GAN}
\small

\begin{block}{What is the Generator?}
The generator $G$ creates \textbf{synthetic data} by transforming
random noise $z$ into an output $G(z)$.

\begin{itemize}
    \item Learns to make fake data look real
    \item Tries to fool the discriminator $D$
    \item Uses noise to produce diverse outputs
\end{itemize}
\end{block}

\end{frame}

\begin{frame}{Generator Training}
\small

\begin{block}{Training \& Update Process}
\begin{columns}[T,onlytextwidth]

\begin{column}{0.48\textwidth}
\textbf{Training Steps}
\begin{itemize}
    \item Sample random noise $z$
    \item Compute fake sample $G(z)$
    \item Get $D(G(z))$ classification
\end{itemize}
\end{column}

\begin{column}{0.48\textwidth}
\textbf{Update Step}
\begin{itemize}
    \item Compute generator loss
    \item Backpropagate through $D \rightarrow G$
    \item Update \textbf{only $G$}
\end{itemize}
\end{column}

\end{columns}
\end{block}

\end{frame}

\begin{frame}{Generator Training: Backpropagation}

\centering
\includegraphics[width=0.9\linewidth,height=0.8\textheight,keepaspectratio]{gan_diagram_generator}

\end{frame}

