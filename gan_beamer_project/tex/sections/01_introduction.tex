% ============================================================
%  Section 1: Introduction
%  NOTE: This file is meant to be \input-ed by main.tex.
%        Do NOT compile it directly — compile main.tex instead.
% ── Colour guard (safe duplicate – only defines if not yet set) ──────────────
\providecolor{cream}{RGB}{250,248,240}
\providecolor{crimson}{RGB}{139,26,26}
\providecolor{darkfoot}{RGB}{60,60,60}
\providecolor{lightcrimson}{RGB}{200,60,60}
\providecolor{blockbg}{RGB}{242,236,224}

% ============================================================
\section{Introduction}
\begin{frame}{What is a GAN?}

% --- Definition ---
\begin{block}{Intuition}
\textbf{Generative Adversarial Networks (GANs)} are generative models that 
learn to create new data instances resembling the training data.
\end{block}

\vspace{0.3em}

% --- Key Points ---
\begin{itemize}\setlength\itemsep{0.25em}
  \item Generate realistic \textbf{images, audio, video}, etc.
\end{itemize}

\vfill

% --- Image ---
\begin{center}
\begin{figure}
\centering
\includegraphics[width=0.75\linewidth, trim=0 6cm 0 6cm, clip]{gan_faces}
\vspace{-0.3cm} % Adjust this value (e.g., -0.2cm to pull it closer)
\caption{GAN-generated human faces}
\label{fig:gan_faces}
\end{figure}
\end{center}

\end{frame}
